\documentclass[11pt]{article}
\usepackage{natbib,mybigpackage}
\usepackage{algorithm}
%\usepackage{program}
%\usepackage{algpseudocode}
\usepackage{algorithmic}
\usepackage{listings}

\def\xbf{\mathbf{x}}
\def\zbf{\mathbf{z}}
\def\xibf{\mathbf{\xi}}

\title{MA - 226 Assignment Report}
\author{
\begin{center}
	\begin{tabular}{c c}
	Ayush Sharma & (150123046)\\
	Shashank Oberoi & (150123047)\\
	Ojasvin Sood & (150123048)\\
	Yugansh Bhatnagar & (150123050)\\
	Remidi Nishanth Reddy & (150123051)\\
	\end{tabular}
\end{center}
}
\date{}

\begin{document}
\titlepage
\newpage
\begin{enumerate}
\item[Q 1.] $F(x)$ be General Exponential Distribution, i.e. $F(x) = (1-e^{- \lambda x})^{\alpha}$.
Generate random numbers from distribution $F_{1}(x) = (1 + \lambda)F(x) - \lambda F^{2}(x)$, where $\lambda \in [-1,1]$, using :
\begin{itemize}
	\item Inverse Transform
	\item Acceptance Rejection
\end{itemize}
\noindent{\textbf{Solution}}
\begin{itemize} 
\item \textbf{Inverse Transform Method.}\\
Generating random number from the given distribution $F_{1}(x)$ using Inverse Transform Method:\\
\begin{equation}
\begin{split}
 & F_{1}(x) = (1 + \lambda)F(x) - \lambda F^{2}(x) = u \text{, where } u \sim \mathcal{U}(0,1).\\
\Rightarrow & \lambda F^{2}(x) - (1 + \lambda)F(x) + u = 0\\
\Rightarrow & F(x) = \frac{(1 + \lambda) - \sqrt{(1 + \lambda)^{2} - 4 \lambda u}}{2 \lambda} = q \hspace{10mm} \text{\{see NOTE\}}\\
\Rightarrow & F(x) = (1-e^{- \lambda x})^{\alpha} = q\\
\Rightarrow & (1-e^{- \lambda x}) = q^{\frac{1}{\alpha}}\\
\Rightarrow & e^{- \lambda x} = (1 - q^{\frac{1}{\alpha}})\\
\Rightarrow & - \lambda x = \log(1 - q^{\frac{1}{\alpha}})\\
\Rightarrow & x = - \frac{1}{\lambda} \log(1 - q^{\frac{1}{\alpha}})\\
\end{split}\nonumber
\end{equation}
NOTE: Here we have only considered $F(x) = \frac{(1 + \lambda) - \sqrt{(1 + \lambda)^{2} - 4 \lambda u}}{2 \lambda}$ and not $F(x) = \frac{(1 + \lambda) + \sqrt{(1 + \lambda)^{2} - 4 \lambda u}}{2 \lambda} $ because 
\begin{equation}
\begin{split}
 &u \sim U(0,1) \Rightarrow u < 1\\
\Rightarrow & (1 + \lambda)^{2} - 4 \lambda u  ~>~ (1 + \lambda)^{2} - 4\lambda \\
\Rightarrow & \sqrt{(1 + \lambda)^{2} - 4 \lambda u} ~>~ |1-\lambda|  \\
\Rightarrow & \sqrt{(1 + \lambda)^{2} - 4 \lambda u} ~>~ 1-\lambda   ~~~~~given~~~~~ \lambda \in [-1,1]\\
\Rightarrow & (1 + \lambda) - \sqrt{(1 + \lambda)^{2} - 4 \lambda u} ~<~ (1+\lambda + \lambda -1)=2\lambda  \\
\Rightarrow & \frac{(1 + \lambda) - \sqrt{(1 + \lambda)^{2} - 4 \lambda u}}{2 \lambda} < 1\\
\Rightarrow & F(x) < 1\\
\Rightarrow & F(x)=  (1-e^{- \lambda x})^{\alpha} ~~and~~ 1-e^{- \lambda x} \leq 1 ~,~ \lambda \geq 0 \\
\end{split}\nonumber
\end{equation}
\begin{algorithm}[H]
\caption{Generating random number from the distribution by inverse transform method.}
\begin{algorithmic}[1]
\STATE Generate $U \sim \mathcal{U}(0,1)$.
\STATE Generate $Q$ from the relation $Q = \frac{(1 + \lambda) - \sqrt{(1 + \lambda)^{2} - 4 \lambda U}}{2 \lambda}$.
\STATE Generate $X$ from the relation $X = - \frac{1}{\lambda} \log(1 - Q^{\frac{1}{\alpha}})$.
\STATE Return $X$.
\end{algorithmic}
\end{algorithm}
%\newpage
\noindent{\textbf{Code for R}}
\begin{lstlisting}
args<-commandArgs(TRUE)

gen_inv <- function(lambda, alpha, n) {
	U <- runif(n, 0, 1);
	Q <- vector(length = n);
	X <- vector(length = n);
	Q <- ((1 + lambda) - sqrt((1 + lambda)^2 - (4 * lambda * U))) / (2 * lambda);
	X <- - log(1 - Q^(1 / alpha)) / lambda;

	pdf("1a.pdf");
	hist(X, breaks = 50, main = "");
	legend('topright', legend = c(paste("lambda =", lambda), paste("alpha =", alpha), paste("sample size =", n)), lty = 0, col = "white", bty = 'n');

	cat("The sample mean and variance, given lambda =", lambda, ", alpha =", alpha, ", and sample size ", n, ", are estimated to be", mean(X), ", and", var(X), ", respectively.\n")
}

lambda = as.numeric(args[1]);
alpha = as.numeric(args[2]);
n = as.integer(args[3]);

set.seed(1);

gen_inv(lambda, alpha, n);
\end{lstlisting}
\newpage
\noindent{\textbf{Results}:}\\
\begin{figure}[H]
	\centering
	\includegraphics[width=0.75\textwidth]{"1a".pdf}
		\caption{Histogram for generated random numbers using Inverse Transform Method}
\end{figure}

The sample mean and variance, given $\lambda$ 0.5, $\alpha$ 2, and sample size 1000, are estimated to be 2.441197, and 3.944559, respectively.
\newpage
\item \textbf{Acceptance Rejection Method.}\\
For generating random number from the distribution $F_{1}(x)$ by acceptance-rejection method, the distribution $F(x)$ is chosen as our candidate distribution.\\
We can write the probability density function of distribution $F_{1}(x)$ as
$$f_{1}(x) = (1 + \lambda)f(x) - \lambda F(x) f(x),$$
where $f(x)$ is the probability density function of distribution $F(x)$.\\
Also, we know that $F(x) = (1-e^{- \lambda x})^{\alpha}$.\\
To apply inverse transform method we equate $$u_{1} = F(y) = (1-e^{- \lambda y})^{\alpha}   \Rightarrow   y = - \frac{1}{\lambda} \log(1 - u_{1}^{\frac{1}{\alpha}}),$$ where $u_{1} \sim \mathcal{U}(0,1)$.\\
We choose $c$ maximizing $$\frac{f_{1}(y)}{f(y)} = (1 + \lambda) - 2 \lambda F(y),$$ where $-1 \leq \lambda \leq 1$ and $0 \leq F(y) \leq 1$.\\
The maximum value attained will be $(1 + |\lambda|)$ which is to be taken as $c$.\\
Here,the rejection test $u_{2} > \frac{f_{1}(y)}{cf(y)}$ can be implemented as $$u_{2} > \frac{(1 + \lambda) - 2 \lambda F(y)}{1 + |\lambda|} = \frac{(1 + \lambda) - 2 \lambda u_{1}}{1 + |\lambda|},$$ where $u_{2} \sim \mathcal{U}(0,1)$.

\begin{algorithm}[H]
\caption{Generating random number from the distribution by acceptance-rejection method.}
\begin{algorithmic}[1]
\STATE Generate $U_1, U_2 \sim \mathcal{U}[0,1]$.
\IF {$U_{2} \leq \frac{(1 + \lambda) - 2 \lambda U_{1}}{1 + |\lambda|}$}
	\STATE $X = - \frac{1}{\lambda} \log(1 - U_{1}^{\frac{1}{\alpha}})$.
	\STATE Return $X$.
\ELSE
	\STATE return to step 1.
\ENDIF
\end{algorithmic}
\end{algorithm}
\newpage
\noindent{\textbf{Code for R}}
\begin{lstlisting}
args<-commandArgs(TRUE)

gen_ar <- function(lambda, alpha, n) {
	X <- vector(length = n);

	for (i in 1:n) {
		repeat {
			u<-runif(2, 0, 1);
			if (u[2] <= ((1+lambda) - (2*lambda*u[1]))/(1 + abs(lambda))) {
				X[i] = - (log(1 - u[1]^(1 / alpha)) / lambda);
				break;
			}
		}
	}

	pdf("1b.pdf");
	hist(X, breaks = 50, main = "");
	legend('topright', legend = c(paste("lambda =", lambda), paste("alpha =", alpha), paste("sample size =", n)), lty = 0, col = "white", bty = 'n');

	cat("The sample mean and variance, given lambda", lambda, ", alpha", alpha, ", and sample size ", n, ", are estimated to be", mean(X), ", and", var(X), ", respectively.\n")
}

lambda = as.numeric(args[1]);
alpha = as.numeric(args[2]);
n = as.integer(args[3]);

set.seed(1);

gen_ar(lambda, alpha, n);
\end{lstlisting}
\newpage
\noindent{\textbf{Results}:}\\
\begin{figure}[H]
	\centering
	\includegraphics[width=0.75\textwidth]{"1b".pdf}
		\caption{Histogram for generated random numbers using Acceptance Rejection Method}
\end{figure}

The sample mean and variance, given $\lambda$ 0.5, $\alpha$ 2, and sample size  1000, are estimated to be 2.43525, and 3.771058, respectively.
\end{itemize}
\newpage
\item[Q 2.] $F(x,y)$ be Bi-variate General Exponential Distribution(BVGE). Generate random numbers from distribution $F_{1}(x,y) = (1 + \lambda)F(x,y) - \lambda F^{2}(x,y)$, where $\lambda \in [-1,1]$.

\noindent{\textbf{Solution}}\\
For generating random number from the distribution $F_{1}(x,y)$ by acceptance-rejection method, the distribution $F(x)$ is chosen as our candidate distribution.\\
We can write the probability density function of distribution $F_{1}(x,y)$ as
$$f_{1}(x,y) = (1 + \lambda)f(x,y) - 4 \lambda F(x,y) f(x,y),$$
where $f(x,y)$ is the probability density function of distribution $F(x,y)$.\\
Also, we know that $F(x,y) = (1-e^{- \lambda x})^{\alpha_1}\cdot(1-e^{- \lambda y})^{\alpha_2}\cdot(1-e^{- \lambda ~z})^{\alpha_3},$ where $z=min(x,y)$.\\
To apply inverse transform method
$$U_{0} \sim GE(\alpha_{1},\lambda) \hspace{10mm} U_{1} \sim GE(\alpha_{2},\lambda) \hspace{10mm} U_{2} \sim GE(\alpha_{3},\lambda)$$
\begin{equation}
\begin{split}
P(max(U_0,U_1)\leq x,max(U_0,U_2) \leq y) & = P(U_0 \leq x, U_1 \leq x, U_0 \leq y, U_2 \leq y) \\
 & = P(U_0\leq min(x,y),U_1 \leq x,U_2 \leq y) \\
 & = P(U_0\leq min(x,y)) \cdot P(U_1 \leq x) \cdot P(U_2 \leq y) \\
 & = (1-e^{- \lambda ~min(x,y)})^{\alpha_0} \cdot (1-e^{- \lambda x})^{\alpha_1} \cdot (1-e^{- \lambda y})^{\alpha_2} \\
\end{split}\nonumber
\end{equation}
And hence if $(X_1,X_2)\sim$ BVGE $(\lambda,\alpha_1,\alpha_2,\alpha_3)$ then the joint CDF of $(X_1,X_2)$ for $x_1 > 0 , x_2 > 0$ is 
$$F_{X_1,X_2}(x_1,x_2) = (1-e^{- \lambda x_1})^{\alpha_1} \cdot (1-e^{- \lambda x_2})^{\alpha_2} \cdot (1-e^{- \lambda z})^{\alpha_3},$$
where $z=min(x_1,x_2)$.\\
We choose $c$ maximizing $$\frac{f_{1}(x,y)}{f(x,y)} = (1 + \lambda) - 4 \lambda F(x,y),$$ where $-1 \leq \lambda \leq 1$ and $0 \leq F(x,y) \leq 1$.\\
The maximum value attained will be $(1 + \lambda)$ which is to be taken as $c$.\\
Here,the rejection test $u > \frac{f_{1}(x,y)}{cf(x,y)}$ can be implemented as $$u > \frac{(1 + \lambda) - 4 \lambda F(x,y)}{1 + \lambda},$$ where $u \sim \mathcal{U}(0,1)$.

\begin{algorithm}[H]
\caption{Generating random number from the distribution by acceptance-rejection method.}
\begin{algorithmic}[1]
\STATE Generate $U_{0} \sim GE(\alpha_{1},\lambda), U_{1} \sim GE(\alpha_{2},\lambda) , U_{2} \sim GE(\alpha_{3},\lambda)$.
\STATE Generate $X = (X_{1}, X_{2})$ from the relation $X_{1} = max~\{U_0,U_1\}$ and $X_2=max~\{U_0,U_2\}$.
\IF {$u \leq 1-\frac{4\lambda F(X)}{1+\lambda} \text{, where } u\sim \mathcal{U}(0,1)$}
	\STATE Return $X$.
\ELSE
	\STATE return to step 1.
\ENDIF
\end{algorithmic}
\end{algorithm}


%\begin{equation}
%\begin{split}
%P(max(U_0,U_1)\leq x_1,max(U_0,U_2) \leq x_2) & = P(U_0\leq x_1,U_1\leq x_1,U_0\leq x_2,U_2 \leq x_2) \\
% & = P(U_0\leq min(x_1,x_2),U_1\leq x_1,U_2 \leq x_2) \\
% & = P(U_0\leq min(x_1,x_2))~.~P(U_1\leq x_1)~.~P(U_2 \leq x_2) \\
% & = (1-e^{- \lambda ~min(x_1,x_2)})^{\alpha_0}~.~(1-e^{- \lambda x_1})^{\alpha_1}~.~(1-e^{- \lambda x_2})^{\alpha_2} \\
%\end{split}\nonumber
%\end{equation}

%And hence if $(X_1,X_2)\sim$  BVGE $(\lambda,\alpha_1,\alpha_2,\alpha_3)$ then the joint CDF of $(X_1,X_2)$ for $x_1 > 0 , x_2 > 0$ is $$F_{X_1,X_2}(x_1,x_2) =  (1-e^{- \lambda ~x_1})^{\alpha_1}~.~(1-e^{- \lambda x_2})^{\alpha_2}~.~(1-e^{- \lambda ~z})^{\alpha_3},$$ where $z=min(x_1,x_2)$.

\noindent{\textbf{Code for R}}
\begin{lstlisting}
args<-commandArgs(TRUE)

library(MASS) #For kde2d

#Function BVGEF Computes Value F(X)
BVGEF <- function(X, lambda, alpha) {
	return ((1 - exp(-lambda * X[1]))^(alpha[2]) * (1 - exp(-lambda * X[2]))^(alpha[3]) * (1 - exp(-lambda * min(X)))^(alpha[1]));
}

#Function GE generates n numbers from generalised exponential
GE <- function(n, lambda, alpha) {
	U <- runif(n,0,1);
	return (- log(1 - U^(1 / alpha)) / lambda);
}

#Function BVGE generates n numbers from Bivariate generalised exponential
BVGE <- function(n, lambda, alpha) {
	bvge <- matrix(0, nrow = n, ncol = 2);

	for(i in 1:n) {
		U <- vector(length = 3);
		U[1]<-GE(1,lambda,alpha[1]);
		U[2]<-GE(1,lambda,alpha[2]);
		U[3]<-GE(1,lambda,alpha[3]);		

		bvge[i,] <- c(max(U[1], U[2]), max(U[1], U[3]));
#		bvge[i,1] <- max(U[1], U[2]);
#		bvge[i,2] <- max(U[1], U[3]);
	}
	return (bvge);
}

#Function TRUNC_BVGE generates n numbers from Truncated Bivariate generalised exponential
TRUNC_BVGE <- function(n, lambda, alpha) {
	trunc_bvge<-matrix(0, nrow = n, ncol = 2);

	for (i in 1:n) {
		repeat {
			u<-runif(1, 0, 1);
			X <- BVGE(1, lambda, alpha);
			F <- BVGEF(X, lambda, alpha);

			if (u <= (1 - (4 * lambda * F)/(1 + lambda))) {
				trunc_bvge[i,] <- X;
				break;
			}
		}
	}

	return(trunc_bvge);
}  

n = as.integer(args[1]);
lambda = as.numeric(args[2]);
#alpha <- vector(length = 3);
alpha <- c(as.numeric(args[3]), as.numeric(args[4]), as.numeric(args[5]));

set.seed(1);

#X <- TRUNC_BVGE(1000,1,0.5,0.6,0.7);
X <- TRUNC_BVGE(n, lambda, alpha);
f <- kde2d(X[,1], X[,2], n = n); # Two dimensional kernel density approximation

cat("The sample means, variances, and covariance, given lambda", lambda, ", alpha", alpha, ", and sample size ", n, ", are estimated to be :","\nmean(X1) =", mean(X[,1]), "\nmean(X2) =", mean(X[,2]), "\nvariance(X1) =", var(X[,1]), "\nvariance(X2) =", var(X[,2]), "\ncorrelation(X1, X2) =", cor(X[,1], X[,2]), "\n");

pdf("2.pdf");
#contour(f, xlab = "X1", ylab = "X2", main = "")
persp(f, xlab = "X1", ylab = "X2", zlab = "Density", main = "", box = TRUE)
legend('topright', legend = c(paste("lambda =", lambda), paste("alpha =", alpha[1], ",", alpha[2], ",", alpha[3]), paste("sample size =", n)), lty = 0, col = "white", bty = 'n');
\end{lstlisting}

\noindent{\textbf{Results}:}\\
\begin{figure}[H]
	\centering
	\includegraphics[width=0.75\textwidth]{"2".pdf}
		\caption{Density Plot for generated bi-variate random numbers using Acceptance Rejection Method}
\end{figure}

The sample means, variances, and covariance, given $\lambda$ 1, $\alpha$ (0.5, 0.6, 0.7), and sample size 100, are estimated to be :\\ 
mean(X1) = 0.593018; \hspace{10mm}
mean(X2) = 0.6326948\\
variance(X1) = 0.5792707; \hspace{10mm}
variance(X2) = 0.5200241\\
correlation(X1, X2) = -0.137354

\end{enumerate}
\end{document}

%#Made by Ayush Sharma#
%#Signed as AShar#